
\documentclass[11pt]{article} 
\usepackage[left=1in,top=1in,right=1in,bottom=1in,nohead]{geometry}
\usepackage{pdfsync}
\usepackage{setspace}
\usepackage{hyperref} % add hyperlinks to citations
\hypersetup{backref,colorlinks=true,citecolor=blue,linkcolor=blue,urlcolor=blue} % add color of citations

%% Call packages that allow you to invoke certain mathematical symbols.
\usepackage{amssymb,amsmath,amsthm}
\usepackage{graphicx, caption, subcaption, listings, color}

\usepackage[sort]{natbib} % [number, square] will let the
\usepackage[font={small,it}, margin=2cm]{caption} % set figure caption
                                % font, size, width.

\DeclareMathOperator*{\argmin}{argmin}
\newcommand{\norm}[1]{\left\lVert #1 \right\rVert}

\begin{document}

% Set the title, author, and date information.
\begin{center}
\noindent
DJM\hfill Possible proof technique \hfill \today \\
%\today \\
\rule{6.5in}{1pt}
\end{center}

Building on PCRandCS (via DW),
\begin{itemize}
\item I'm thinking it will be better to do this under the model in the
  paper. This means $U_{p-d}=0\Rightarrow R_d=0$, but no harm in
  keeping it around for now.
\item Eq.3 $\rightarrow$ Eq.4, $\tilde{U}_d$ became $U_d$, so
  $V\Lambda$ needs to be $V_d\Lambda_d$ in Eq.4 (fixed in Eq.5).
\item Eq.4 $\rightarrow$ Eq.5, factoring out $U_d^\top Y$, the second
  half got missed. So starting from Eq.6, we should have
  \begin{align}
  &=\norm{U_d(F)\Lambda_d(F)^{-1}U_d(F)^\top V_d\Lambda_d -
    V_d\Lambda_d^{-1}} M_d + R_d \mbox{ (dropping $R_d$ now)}\\
  &\leq \norm{U_d(F)\Lambda_d(F)^{-1}U_d(F)^\top V_d\Lambda_d -
    U_d(F)\Lambda_d(F)^{-1}\Lambda_d}M_d\\
  &\quad + \norm{U_d(F)\Lambda_d(F)^{-1}\Lambda_d -
    V_d\Lambda_d^{-1}} M_d\\
  &\leq \norm{U_d(F)\Lambda_d(F)^{-1}}\norm{U_d(F)^\top
    V_d-I}\norm{\Lambda_d} M_d\\
  &\quad +\norm{U_d(F)\Lambda_d(F)^{-1/2}\Lambda_d(F)^{-1/2}\Lambda_d -
    V_d\Lambda_d^{-1}} M_d\\
  &\leq \norm{U_d(F)\Lambda_d(F)^{-1}}\norm{U_d(F)^\top
    V_d-I}\norm{\Lambda_d} M_d\\
  &\quad
    +\norm{U_d(F)\Lambda_d(F)^{-1/2}}\norm{\Lambda_d(F)^{-1/2}\Lambda_d
    - I}M_d + \norm{U_d(F)\Lambda_d(F)^{-1/2}-
    V_d\Lambda_d^{-1}} M_d
  \end{align}
\item Is there a relationship between $\norm{\Lambda_d}$ and
  $\norm{\Lambda_d(F)}$? This would be nice.
\item $M_d$ seems like it will be a pain: $\Theta(n)$.
\item My thinking (up to now) had been to mimic Paul, Bair, et.\ al:
  \begin{enumerate}
  \item Show that $\norm{\sin(\mathcal{E},\mathcal{F})}$ is small
    where $\mathcal{E}$ is the span of $V_d$ and $\mathcal{F}$ is the
    span of $U_d(F)$.
  \item Show that $\norm{\Lambda(F)_d-\Lambda_d}$ is small.
  \item See whether this gives anything about $\hat{\beta}_d$.
  \end{enumerate}
\item For the first step, this would amount to examining a function of
  $V_dV_d^\top-U_d(F)U_d(F)^\top$. I was thinking with Lemma 4.2 or
  Corollary 4.1 in Lei and Vu's sparse PCA paper. Although, this again
  is just a different way of measuring the approximation accuracy of
  $U_d(F)$.
\item My thoughts on the target journal here is JCGS. To that end, I
  think we need some or all of the following:
  \begin{enumerate}
  \item Minor theoretical contributions along the lines above. Get as
    far as we can before it gets painful, likely under strong assumptions.
  \item Do the Nystrom version as well. (Already done in simulations,
    it's a bit worse, though not terrible)
  \item Implement GLMs.
  \end{enumerate}


\end{itemize}





\end{document}
